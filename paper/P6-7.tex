\documentclass[11pt,twoside]{article}

% Do not use packages other than asp2014. 
\usepackage{asp2014}

\aspSuppressVolSlug
\resetcounters

\bibliographystyle{asp2014}

\markboth{Deil at al.}{gamma-cat: an open data collection and source catalog for gamma-ray astronomy}

\newcommand{\gammacat}{\texttt{gamma-cat}}
\newcommand{\gammapy}{\texttt{Gammapy}}
\newcommand{\gadf}{\texttt{gamm-astro-data-formats}}
\newcommand{\gammasky}{\texttt{gamma-sky.net}}
\newcommand{\tevcat}{\texttt{TeVCat}}
\newcommand{\tegev}{\texttt{TeGeV}}

\newcommand{\nsources}{163}
\newcommand{\npapers}{32}

\newcommand{\yaml}{\texttt{YAML}}
\newcommand{\ecsv}{\texttt{ECSV}}
\newcommand{\fits}{\texttt{FITS}}
\newcommand{\votable}{\texttt{VOTABLE}}
\newcommand{\json}{\texttt{JSON}}
\newcommand{\ascii}{\texttt{ASCII}}

\newcommand{\desylc}{\url{https://astro.desy.de/gamma_astronomy/magic/projects/light_curve_archive/}}
\newcommand{\magicurl}{\url{http://vobs.magic.pic.es/fits/}}
\newcommand{\hesscat}{\url{https://www.mpi-hd.mpg.de/hfm/HESS/pages/home/sources/}}
\newcommand{\obspm}{\url{http://hess.obspm.fr/}}
\newcommand{\veritasdata}{\url{http://veritas.sao.arizona.edu/veritas-science/veritas-results-mainmenu-72}}
\newcommand{\hawcdata}{\url{http://www.hawc-observatory.org/publications/}}
\newcommand{\tevcaturl}{\url{http://tevcat.uchicago.edu/}}
\newcommand{\tgevcaturl}{\url{http://www.asdc.asi.it/tgevcat/}}
\newcommand{\gammacatgh}{\url{https://github.com/gammapy/gamma-cat}}
\newcommand{\astrocats}{\url{https://astrocats.space/}}
\newcommand{\pyads}{\url{https://github.com/andycasey/ads}}
\newcommand{\gadfurl}{\url{http://gamma-astro-data-formats.readthedocs.io/}}
\newcommand{\gammapycat}{\url{http://docs.gammapy.org/en/latest/catalog/}}
\newcommand{\whsp}{\url{http://www.asdc.asi.it/2whsp/}}
\newcommand{\fermiurl}{\url{http://fermi.gsfc.nasa.gov/ssc/data/access/lat/}}
\newcommand{\simbadhess}{\url{http://simbad.u-strasbg.fr/simbad/sim-id?NbIdent=cat&Ident=HESS}}

\begin{document}

\title{gamma-cat: an open data collection and source catalog for gamma-ray astronomy}

\author{Christoph~Deil,$^1$
Arjun~Voruganti,$^1$
Axel~Donath,$^1$
Johannes~King,$^1$
Catherine~Boisson,$^2$
Konstancja~Satalecka,$^3$
and Matthias~Wegen$^3$
\affil{$^1$MPIK, Heidelberg, Germany; \email{Christoph.Deil@mpi-hd.mpg.de}}
\affil{$^2$LUTH, Observatoire de Paris, Meudon, France}
\affil{$^3$DESY, Zeuthen, Germany}
}

% This section is for ADS Processing.  There must be one line per author.
\paperauthor{Christoph~Deil}{Christoph.Deil@mpi-hd.mpg.de}{}{MPIK}{}{Heidelberg}{}{}{Germany}
\paperauthor{Arjun~Voruganti}{arjun.voruganti@gmail.com}{}{MPIK}{}{Heidelberg}{}{}{Germany}
\paperauthor{Axel~Donath}{Axel.Donath@mpi-hd.mpg.de}{}{MPIK}{}{Heidelberg}{}{}{Germany}
\paperauthor{Johannes~King}{Johannes.King@mpi-hd.mpg.de}{}{MPIK}{}{Heidelberg}{}{}{Germany}
\paperauthor{Catherine~Boisson}{Catherine.Boisson@mpi-hd.mpg.de}{}{LUTH}{}{Meudon}{}{}{France}
\paperauthor{Konstancja~Satalecka}{konstancja.satalecka@desy.de}{}{DESY}{}{Zeuthen}{}{}{Germany}
\paperauthor{Matthias~Wegen}{matthias.wegen@desy.de}{}{DESY}{}{Zeuthen}{}{}{Germany}

\begin{abstract} \gammacat\ (\gammacatgh) is an open data collection and source
catalog for TeV gamma-ray astronomy. Currently data from \npapers~papers is
available, and the catalog contains \nsources~sources. Data is input using the
hierarchical, human- and machine-readable \yaml\ format and the tabular \ecsv\
text formats, processed using Python scripts into an as-uniform form as
possible. The data can be browsed on the \gammasky~web page, analyzed using
\gammapy, or fully downloaded in \fits\ and other formats and used in whatever
way the user likes. Data is collected in a git repository on Github, providing
transparency, version control as well as simple maintenance and contribution
workflow. This data repository was started in August 2016, the data collection
as well as the specification of the input and output formats is work in
progress. Here we present the project for the first time, and discuss it's
context, implementation, status, plans as well as some possible use cases for
science analysis. \end{abstract}

\section{Introduction}

The first cosmic TeV gamma-ray source detected from the ground was the Crab
nebula in 1989. Since then, TeV astronomy has seen rapid growth. As of October
2016, \nsources~sources have been detected (see Figure~\ref{fig:allsky}).
Information on the position, morphology, spectra and lightcurves of these
sources have been published by HEGRA, H.E.S.S., VERITAS, MAGIC, HAWC and other
telescopes in a few hundred individual papers. There is a need for a
machine-readable, curated, up-to-date collection of TeV data and a TeV source
catalog. Here we give an overview of other TeV gamma-ray data collections and
catalogs in Section~\ref{sec:others}, present the new project \gammacat\ in
Section~\ref{sec:gammacat} and conclude in Section~\ref{sec:conclusions}.

\articlefigure[width=1.0\textwidth]{figures/allsky.eps}{fig:allsky}{TeV
gamma-ray sources from \gammacat\ (white circles, \nsources~sources, status
October~2016). The image (counts, smoothed with a Gaussian of width
$\sigma=0.3$~deg) shows the gamma-ray sky above 50~GeV using the Fermi-LAT 2FHL
dataset \citep{fermi-2fhl}.}

\section{Other TeV gamma-ray data collections and catalogs}
\label{sec:others}

For a given paper, the measurements on TeV gamma-ray sources are often not
available in a machine-readable format. Typically the source position,
morphology and spectral model are fitted to the data, and the results of that
fit are given in the paper text. Sometimes, spectral points or lightcurve points
are given in \ascii\ or \fits\ format on some webpage, or can be obtained from
the corresponding author. For current TeV instruments, the following data is
available: A H.E.S.S. source list is available at \hesscat. Spectra and
lightcurves for most H.E.S.S. papers are listed as text on per-paper webpages, a
curated collection for blazars in machine-readable format is available \obspm.
For VERITAS, lightcurves and spectra in FITS format can be found at
\veritasdata, for MAGIC at \magicurl. A collection of AGN lightcurve data is
available at \desylc, which is an evolution of the dataset described in
\citet{lc}.

Several other projects to collect available gamma-ray data from several
telescopes exist. For Fermi-LAT, many catalogs and other high-level data
products are available at \fermiurl, and this information is also available via
the HEASARC and ADSC archives. Here we focus on TeV data and not discuss
Fermi-LAT data archives further for now. \tevcat\ is an online TeV source
catalog \citep{tevcat}, available at \tevcaturl. \tevcat\ is not available for
download, and the terms and conditions page explicitly forbids scraping the
\tevcat\ web page to download a copy of the data. \tevcat\ does not contain
spectral points or lightcurve or image data.

\tegev\ is a TeV source catalog \citep{tgevcat} available at \tgevcaturl.
\tegev\ is part of a larger collection of multi-wavelength data and web-based
tools to browse and analyze the data, and does include TeV spectra and
lightcurves. The \tegev\ is available for download in CSV format, downloading
all data (including spectral points and lightcurves) would require scraping the
website or contacting the maintainers. 2WHSP \citep{whsp}, available at \whsp,
is another recent ASDC catalog, focusing on gamma-ray blazars. SIMBAD and Vizier at CDS are collections of all astronomical objects. They do contain information on TeV gamma-ray sources, e.g. a list of H.E.S.S. sources known to SIMBAD is at \simbadhess. However, with SIMBAD it's currently not easy to get a list of gamma-ray sources, and they don't have detailed information on the gamma-ray sources, e.g. morphology information, spectral point or lightcurve data.

Given that these gamma-ray data collection exist, the question ``why start a new one?'' needs to be answered. We will first present \gammacat\ in the next section, and then address this question in the conclusions.

\section{\gammacat}
\label{sec:gammacat}

\subsection{An open TeV data collection and catalog}

\gammacat\ (\gammacatgh) is both a \emph{TeV data collection} and a \emph{TeV
source catalog}. The data collection consists of measurements from papers in
machine-readable form. From this collection, a TeV source catalog is derived as
a higher-level data product. The source catalog is somewhat subjective and
doesn't contain all data from the collection. For a given source, usually each
paper contains a set of measurements (e.g. source position, morphology,
spectrum) and for the catalog, the ``best'' available parameters are chosen.
Sometimes, especially for sources in the inner Galactic plane region where the
source density is high, it can even happen that a later higher-resolution image
reveals that what was previously thought of as one ``source'' really consists of
multiple ``sources''. Extra-galactic sources usually don't have this issue of
source confusion, but the most common source class, active galactic nuclei
(AGN), are often variable in flux and spectral shape, implying that no single
measurement can be the ``best''. The source identification and association
status can change over time, as new potential counterparts (e.g. pulsars,
supernova remnants, AGN, \ldots) are discovered in multi-wavelength data. For
these reasons, we consider the \emph{TeV data collection} aspect of primary
importance --- \gammacat\ is a service to the astronomical community where
information from hundreds of individual papers has been collected into a
machine-readable form that can be easily queried and used. The \gammacat\
catalog is a secondary data product that provides a simplified summary of the
available information on TeV gamma-ray sources.

The \gammacat\ scope is not completely decided. Currently the focus is on
collecting the published measurements for source position, morphology, spectrum
and lightcurves for TeV gamma-ray sources, mostly from the past decade and the
H.E.S.S., VERITAS, MAGIC and HAWC telescopes. However, if we receive
contributions of similar datasets in machine-readable formats that are not
readily available elsewhere, we will consider adding them (but not exposing them
in the gamma-ray source catalog). Examples that come to mind are: Fermi-LAT
gamma-ray source measurements from individual papers in the GeV--TeV energy
range, TeV gamma-ray diffuse emission measurements or models, cosmic ray
electron spectra. Whether \gammacat\ will remain solely focused on TeV gamma-ray
sources, or will expand to become more of a ``TeV gamma-ray very-high-level
datasets'' repository remains to be seen. Another choice we will have make is
how to expose or merge the information from upcoming TeV source catalogs by
H.E.S.S. and HAWC.

\subsection{Guiding principles and implementation}

The creation of \gammacat\ was inspired by the other open astronomy catalogs at
\astrocats. In the paper describing the \emph{Open supernova catalog},
\citet{sne-cat} explain their guiding principles and describe the resulting
implementation. \gammacat\ is very similar, and in this section we briefly
summarize our guiding principles and implementation.

\emph{\gammacat\ data is fully open-access}. Anyone can download all available
data as a \texttt{zip} file, or clone the git repository to obtain the full
version history of \gammacat\ and use it however they like (except for an
attribution request, which we state in our terms of use). Currently the size is
very small, roughly 1~MB. We expect this to grow over time, but to stay within
the limit of roughly 1~GB that comfortably fits in a single git repository. If
large data files were added at some point (e.g. FITS images), extra repositories
or other data stores could be used.

\emph{\gammacat\ is set up for fully open collaboration}. The input data
collection is stored using the hierarchical, human- and machine-readable \yaml\
format and the tabular \ecsv\ text formats. The output data products (source
catalog, spectra, lightcurves) are created using Python scripts (just run one
command \texttt{python make.py all}). All scripts and Python packages we use are
open-source: PyYAML (\url{http://pyyaml.org/}), Astropy \citep{astropy}, Gammapy
\citep{gammapy} and ADS (\pyads). By making the data, tools and process fully
open, we hope to create a community-driven project with a large scope and
lifetime. Using a git repo and the simple ``collection of text files as
database'' implementation means that the project is not tied to a given data
center, website or researcher. Github pull requests provide a simple way to
accept and review contributions without giving new people write access from the
start.

\subsection{Data formats, products, website, Python tools}

The data products offered for \gammacat\ are a source catalog as well as
per-paper data. The source catalog is a flat table (available in \ecsv, \fits\
and \votable\ format) that contains one row per source with the most important
information listed in a simple format. The catalog does not contain spectral
points or lightcurve points or measurements from all papers for a given source.
The full data available in \gammacat\ is given as a collection of files that are
generated based on collected input \yaml\ and \ecsv\ data, and derived
information from the Python scripts. E.g. we compute the integral flux above
1~TeV according to the best-fit spectral model, even if that number wasn't given
in the paper. Currently we are using \json\ for hierarchical data and \ecsv\ for
tabular output data. Other formats (e.g. \yaml\ or \fits\ and \votable) could
easily be generated as well, this will likely change as we receive feedback on
which formats are preferred by the community. Development of the data format
specifications (e.g. for spectral points or lightcurves) is work in progress and
happens at \gadfurl\ \citep{open-gamma}, the formats we are using now should be
considered unstable prototypes. We have started to develop a Python package,
\texttt{gammapy.catalog} (\gammapycat, \citet{gammapy}), that allows easy
querying of \gammacat\ and working with the data, e.g. plotting spectra or
lightcurves or analyzing them, or generating a TeV sky model from the catalog.

% \subsection{Application example -- Mrk 421 lightcurve}

% Mention multi-mission analysis with Sherpa \citep{sherpa}?

% \subsection{Spectra}
% 
% Example spectrum, see Figure~\ref{fig:spec}.
% 
% \articlefigure[width=.8\textwidth]{figures/spec.eps}{fig:spec}{Spectrum example. Crab nebula.}
% 
% \subsection{Light curves}

% Example lightcurve, see Figure~\ref{fig:lc}.
% 
% \articlefigure[width=1.0\textwidth]{figures/lightcurve.eps}{fig:lc}{TeV gamma-ray lightcurve of Mrk 421. Figure~from~\citet{lc}. This is an example of a dataset that will be available via \gammacat\ soon. At this time, it is available at \desylc.}

\section{Conclusions}
\label{sec:conclusions}

% Note that none of the other TeV gamma-ray data collections and catalogs we are
% aware of (listed in Section~\ref{sec:others}) is fully open-access and open for
% collaboration like \gammacat. We invite everyone to ...

% TODO: should we have such comments here?
% \tevcat\ is usually very up-to-date with newly detected TeV sources and papers, \tegev\ is currently at version 2 from July 2015. In the next 

% The release of the H.E.S.S. Galactic plane survey catalog and the first all-sky
% HAWC catalog is expected soon.

\gammacat\ is a TeV gamma-ray astronomy data collection and source catalog.
Compared to previous similar projects, \gammacat\ stands out be being fully
open-access and set up in a way that makes maintenance and community
contributions easy, following the lead of other open astronomy catalogs from
\astrocats. None of the other TeV data collections can easily be downloaded in
full, but often this is what users want, to e.g. browse, plot or analyze the
data with whatever tools they like. Merging of some previous TeV data
collections (e.g. the light curve archive as DESY) as well as communication
about possible collaboration with the other project is ongoing. \gammacat\ is
being developed in close collaboration with ongoing efforts to develop gamma-ray
data format standards (\gadf), open-source science tools (\gammapy) as well as a
website dedicated to the gamma-ray sky (\gammasky).

\gammacat\ lets you quickly access all published information for all TeV
sources, ready for data exploration or analysis. We hope that this will be a
useful resource for daily work as well as the basis for novel studies involving
the extensive set of available archival gamma-ray data. Concretely we propose
\gammacat\ to be used as one of the inputs to build the sky model for the
planned CTA data challenge. The scope of \gammacat\ is not decided yet, we will
have to see if someone wants to add Fermi-LAT source information, HAWC and
H.E.S.S. source catalog information, or other information. Feedback and
contributions to \gammacat\ are highly welcome!

% \clearpage % To force this stuff to happen by this point in the text, otherwise these will probably end up after the references.

\acknowledgements We thank Fabrizio Lucarelli and Gernot Maier for comments
about \gammacat. This research has made use of NASA's Astrophysics Data System
Bibliographic Services, the SIMBAD database, operated at CDS, Strasbourg,
France, the TeVCat online source catalog and the TeGeV catalog at ASDC, as well
as the following astronomy Python packages: Astropy, Gammapy, ads.

\bibliography{P6-7}

\end{document}
