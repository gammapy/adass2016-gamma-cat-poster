\documentclass[11pt,twoside]{article}

% Do not use packages other than asp2014. 
\usepackage{asp2014}

\aspSuppressVolSlug
\resetcounters

\bibliographystyle{asp2014}

\markboth{Deil at al.}{An open catalog for TeV gamma-ray astronomy}

\begin{document}

\title{An open catalog for TeV gamma-ray astronomy}

\author{Christoph~Deil,$^1$
Arjun~Voruganti,$^1$
Axel~Donath,$^1$
Johannes~King,$^1$
Catherine~Boisson,$^2$
Konstancja~Satalecka,$^3$
and Matthias~Wegen$^3$
\affil{$^1$MPIK, Heidelberg, Germany\email{Christoph.Deil@mpi-hd.mpg.de}}
\affil{$^2$LUTH, Observatoire de Paris, Meudon, France}
\affil{$^3$DESY, Zeuthen, Germany}
}

% This section is for ADS Processing.  There must be one line per author.
\paperauthor{Christoph~Deil}{Christoph.Deil@mpi-hd.mpg.de}{}{MPIK}{}{Heidelberg}{}{}{Germany}
\paperauthor{Arjun~Voruganti}{arjun.voruganti@gmail.com}{}{MPIK}{}{Heidelberg}{}{}{Germany}
\paperauthor{Axel~Donath}{Axel.Donath@mpi-hd.mpg.de}{}{MPIK}{}{Heidelberg}{}{}{Germany}
\paperauthor{Johannes~King}{Johannes.King@mpi-hd.mpg.de}{}{MPIK}{}{Heidelberg}{}{}{Germany}
\paperauthor{Catherine~Boisson}{Catherine.Boisson@mpi-hd.mpg.de}{}{LUTH}{}{Meudon}{}{}{France}
\paperauthor{Konstancja~Satalecka}{konstancja.satalecka@desy.de}{}{DESY}{}{Zeuthen}{}{}{Germany}
\paperauthor{Matthias~Wegen}{matthias.wegen@desy.de}{}{DESY}{}{Zeuthen}{}{}{Germany}

\begin{abstract}

The first cosmic TeV gamma-ray source detected from the ground was the Crab
nebula in 1989. Since then, TeV astronomy has seen rapid growth. By now, over
160 sources have been detected. Measurements of source position, morphology,
spectrum and sometimes lightcurves have been published, mostly in individual
papers. Often the source parameters are not given in a machine-readable format,
and even if they are, there is no common data format.

We present an effort to collect the available data on TeV sources, and curate it
into an as-uniform and as-complete as possible form, and have it freely
available for download at \url{https://github.com/gammapy/gamma-cat}. This
poster presents the project idea and status, as well as its technical
implementation, which includes YAML, ECSV, JSON and FITS files and Python
scripts using Gammapy (\url{http://gammapy.org}), and several other Python
packages. A web front-end to browse this TeV source catalog and other gamma-ray
and multi-wavelength data is under development at \url{http://gamma-sky.net}.

\end{abstract}

\section{Introduction}

Multi-mission analysis with Sherpa \citep{sherpa}

Other projects:

\begin{itemize}
\item \citet{sne-cat} and \url{https://astrocats.space/}
\item \citet{tgevcat} and \url{http://www.asdc.asi.it/tgevcat/}
\item \citet{tevcat} and \url{http://tevcat.uchicago.edu/}
\end{itemize}

Data:

\begin{itemize}
\item Open gamma formats: \citet{open-gamma} and \url{http://gamma-astro-data-formats.readthedocs.io/}
\item Lightcurves: \citet{lc} and \url{https://astro.desy.de/gamma_astronomy/magic/projects/light_curve_archive/}
\item H.E.S.S. data: \url{http://hess.obspm.fr/} and \url{https://www.mpi-hd.mpg.de/hfm/HESS/pages/home/sources/}
\item VERITAS data: \url{http://veritas.sao.arizona.edu/veritas-science/veritas-results-mainmenu-72}
\item MAGIC data: \url{http://vobs.magic.pic.es/fits/}
\end{itemize}

Tools:

\begin{itemize}
\item PyYAML: \url{http://pyyaml.org/}
\item Astropy: \citet{astropy}
\item Gammapy: \citet{gammapy}
\item \url{https://github.com/andycasey/ads}
\end{itemize}

\section{Usage}

tbd

\section{Implementation}

tbd

\section{Examples}


TeV sources, see Figure~\ref{fig:allsky}.

\articlefigure{figures/allsky.eps}{fig:allsky}{TeV sources (using 2FHL Fermi-LAT high-energy image as background).}


\subsection{Light curve}

Example lightcurve, see Figure~\ref{fig:lc}.

\articlefigure{figures/lightcurve.eps}{fig:lc}{Lightcurve example.}

\section{Conclusions}

Remember: gammacat this is useful and awesome.

\clearpage % To force this stuff to happen by this point in the text, otherwise these will probably end up after the references.

\acknowledgements We thank Fabrizio Lucarelli, Gernot Maier, Konrad Bernl\"ohr and Tarek Hassan for useful discussions or feedback on gamma-cat.
This research has made use of NASA's Astrophysics Data System Bibliographic Services, the SIMBAD database, operated at CDS, Strasbourg, France, the TeVCat online source catalog and the TeGeV catalog at ASDC, as well as the following astronomy Python packages: Astropy, Gammapy, ads.

\bibliography{P6-7}

\end{document}
