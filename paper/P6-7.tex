\documentclass[11pt,twoside]{article}

% Do not use packages other than asp2014. 
\usepackage{asp2014}

\aspSuppressVolSlug
\resetcounters

\bibliographystyle{asp2014}

\markboth{Deil at al.}{An open catalog for TeV gamma-ray astronomy}

\newcommand{\gammacat}{\texttt{gamma-cat}}
\newcommand{\gammapy}{\texttt{Gammapy}}
\newcommand{\gadf}{\texttt{gamm-astro-data-formats}}
\newcommand{\gammasky}{\texttt{gamma-sky.net}}
\newcommand{\tevcat}{\texttt{TeVCat}}
\newcommand{\tegev}{\texttt{TeGeV}}

\newcommand{\nsources}{TODO}
\newcommand{\npapers}{TODO}

\newcommand{\yaml}{\texttt{YAML}}
\newcommand{\ecsv}{\texttt{ECSV}}
\newcommand{\fits}{\texttt{FITS}}
\newcommand{\votable}{\texttt{VOTABLE}}
\newcommand{\json}{\texttt{JSON}}

\begin{document}

\title{An open catalog for TeV gamma-ray astronomy}

\author{Christoph~Deil,$^1$
Arjun~Voruganti,$^1$
Axel~Donath,$^1$
Johannes~King,$^1$
Catherine~Boisson,$^2$
Konstancja~Satalecka,$^3$
and Matthias~Wegen$^3$
\affil{$^1$MPIK, Heidelberg, Germany; \email{Christoph.Deil@mpi-hd.mpg.de}}
\affil{$^2$LUTH, Observatoire de Paris, Meudon, France}
\affil{$^3$DESY, Zeuthen, Germany}
}

% This section is for ADS Processing.  There must be one line per author.
\paperauthor{Christoph~Deil}{Christoph.Deil@mpi-hd.mpg.de}{}{MPIK}{}{Heidelberg}{}{}{Germany}
\paperauthor{Arjun~Voruganti}{arjun.voruganti@gmail.com}{}{MPIK}{}{Heidelberg}{}{}{Germany}
\paperauthor{Axel~Donath}{Axel.Donath@mpi-hd.mpg.de}{}{MPIK}{}{Heidelberg}{}{}{Germany}
\paperauthor{Johannes~King}{Johannes.King@mpi-hd.mpg.de}{}{MPIK}{}{Heidelberg}{}{}{Germany}
\paperauthor{Catherine~Boisson}{Catherine.Boisson@mpi-hd.mpg.de}{}{LUTH}{}{Meudon}{}{}{France}
\paperauthor{Konstancja~Satalecka}{konstancja.satalecka@desy.de}{}{DESY}{}{Zeuthen}{}{}{Germany}
\paperauthor{Matthias~Wegen}{matthias.wegen@desy.de}{}{DESY}{}{Zeuthen}{}{}{Germany}

\begin{abstract}

We present the \gammacat, an online data collection and source catalog for TeV
gamma-ray astronomy. Currently data from \npapers~papers is available, and the
catalog contains \nsources~sources. Data is input using the hierarchical, human-
and machine-readable \yaml\ format and the tabular \ecsv\ text formats,
processed using Python scripts into an as-uniform form as possible. The data can
be browsed on the \gammasky~web page, analyzed using \gammapy, or fully
downloaded in \fits\ and other formats and used in whatever way the user likes.
Data is collected in a git repository on Github, providing transparency, version
control as well as simple maintenance and contribution workflow. This data
repository was started in August 2016, the data collection as well as the
specification of the input and output formats is work in progress. Here we
present the project for the first time, and discuss it's context,
implementation, status, plans as well as some possible use cases for science
analysis.

\end{abstract}

\section{Introduction}

The first cosmic TeV gamma-ray source detected from the ground was the Crab
nebula in 1989. Since then, TeV astronomy has seen rapid growth. By now,
\nsources~sources (status: October~2016) have been detected (see
Figure~\ref{fig:allsky}). 
We start with a review of previous efforts to create TeV data collections, and then in the next section describe the new project presented here for the first time: \gammacat.

\articlefigure[width=1.0\textwidth]{figures/allsky.eps}{fig:allsky}{TeV gamma-ray sources from \gammacat\ (white circles, \nsources~sources, status October~2016). The image (counts, smoothed with a Gaussian of width $\sigma=0.3$~deg) shows the gamma-ray sky above 50~GeV using the Fermi-LAT 2FHL dataset \citep{fermi-2fhl}.}

Measurements of source position, morphology, spectrum and sometimes lightcurves
have been published, mostly in individual papers. Often the measurements are not
given in a machine-readable format, sometimes it is given in ASCII or FITS
formats. There have been several prior efforts to collect and curate the
available TeV gamma-ray data. A H.E.S.S. source list is available in HTML
(Hypertext markup language) and CSV (character-separated values) formats at
\url{https://www.mpi-hd.mpg.de/hfm/HESS/pages/home/sources/}, some spectra and
lightcurves of blazars measured by H.E.S.S. in FITS (Flexible Image Transport
system) and VOTable (Virtual Observatory table) formats at
\url{http://hess.obspm.fr/}. Some VERITAS images, lightcurves and spectra in
FITS format are available at
\url{http://veritas.sao.arizona.edu/veritas-science/veritas-results-mainmenu-72}.
MAGIC at \url{http://vobs.magic.pic.es/fits/}. A light curve archive with FITS
and VOTABLE data for some blazars is available at
\url{https://astro.desy.de/gamma_astronomy/magic/projects/light_curve_archive/},
which is an evolution of the data format and dataset described in \citet{lc}.
HAWC publications are listed at
\url{http://www.hawc-observatory.org/publications/}, as far as we know, no HAWC
data in machine-readable format is available for download

% TODO: should we mention HESS HGPS and HAWC first source catalogs?
% TODO: Contrast this with the situation for Fermi-LAT, where many catalogs are available? (but fore individual source publications the situation is similar as for TeV -> no collection of data in machine-readable format available).

An online TeV source catalog (as of October 2016) is available at \url{http://tevcat.uchicago.edu/} \citep{tevcat}. \tevcat\ is not available for download, and the terms and conditions page explicitly forbids systematic download of the data, e.g. by scraping the web page. \tevcat\ does not contain spectral points or lightcurve or image data.
The \tegev\ catalog is available at \url{http://www.asdc.asi.it/tgevcat/} \citet{tgevcat}. \tegev\ is a larger collection of data compared to \tevcat, it does include spectral points and lightcurves. \tegev is part of a larger collection of multi-wavelength data (including e.g. X-ray and GeV gamma-ray data) and web-based tools to browse and analyse the data (e.g. the ASDC data explorer and SED builder). The \tegev data is available for download in CSV format, downloading all the data (including spectral points and lightcurves) would require scraping the website. 

% TODO: should we have such comments here?
% \tevcat\ is usually very up-to-date with newly detected TeV sources and papers, \tegev\ is currently at version 2 from July 2015. In the next 


\section{\gammacat\ -- Open TeV data collection and catalog}

In this section we describe \gammacat\
(\url{https://github.com/gammapy/gamma-cat}), which is both a \emph{TeV data
collection} and a \emph{TeV source catalog}. The data collection consists of
measurements from papers in machine-readable form. From this collection, a TeV
source catalog is derived as a higher-level data product. The source catalog is
somewhat subjective and doesn't contain all data from the collection. For a
given source, usually each paper contains a set of measurements (e.g. source
position, morphology, spectrum) and for the catalog, ``the best available''
parameters are chosen. Sometimes, especially for sources in the inner Galactic
plane region where the source density is high, it can even happen that a later
higher-resolution image reveals that what was previously thought of as one
``source'' really consists of multiple, not clearly distinguishable ``sources''.
Extra-galactic sources usually don't have this issue of source confusion, but
the most common source class, active galactic nuclei (AGN), are often variable
in flux and spectral shape, implying that no single measurement can be ``the
best''. The source identification and association status can change over time,
as new potential counterparts (e.g. pulsars, supernova remnants, AGN, \ldots)
are discovered in multi-wavelength data. For these reasons, we consider the \emph{TeV data collection} aspect of primary importance --- \gammacat\ is a service to the astronomical community where information from hundreds of individual papers has been collected into a machine-readable form that can be easily queried and used. The \gammacat\ catalog is a secondary data product that provides a simplified summary of the available information on TeV gamma-ray sources.

\subsection{Guiding principles}

The following guiding principles motivated the creation of this new project, and partly it's implementation.

\begin{itemize}
\item \gammacat\ is fully open-access. Everyone can download all available data.
\item \gammacat\ is fully transparent. For every measurement in the catalog, it shall be possible to identify the origin (usually a paper).
\item \gammacat\ data is mostly stored in text files. Changes to \gammacat are under version control.
\item \gammacat\ is fully open-source. The scripts
\end{itemize}


Note: This project is a copycat of \citet{sne-cat} and \url{https://astrocats.space/}, for TeV.

TODO: discusss terms of use and attribution here?

\begin{itemize}
\item Open gamma formats: \citet{open-gamma} and \url{http://gamma-astro-data-formats.readthedocs.io/}
\end{itemize}

TODO: discuss project scope

% We present an effort to collect the available data on TeV sources, and curate it
% into an as-uniform and as-complete as possible form, and have it freely
% available for download at \url{https://github.com/gammapy/gamma-cat}. This
% poster presents the project idea and status, as well as its technical
% implementation, which includes YAML, ECSV, JSON and FITS files and Python
% scripts using Gammapy (\url{http://gammapy.org}), and several other Python
% packages. A web front-end to browse this TeV source catalog and other gamma-ray
% and multi-wavelength data is under development at \url{http://gamma-sky.net}.

\subsection{Implementation}

Tools:

\begin{itemize}
\item PyYAML: \url{http://pyyaml.org/}
\item Astropy: \citet{astropy}
\item Gammapy: \citet{gammapy}
\item \url{https://github.com/andycasey/ads}
Multi-mission analysis with Sherpa \citep{sherpa}
\end{itemize}


\section{Application examples}

\subsection{Spectra}

Example spectrum, see Figure~\ref{fig:spec}.

\articlefigure[width=.8\textwidth]{figures/spec.eps}{fig:spec}{Spectrum example. Crab nebula.}

\subsection{Light curves}

Example lightcurve, see Figure~\ref{fig:lc}.

\articlefigure{figures/lightcurve.eps}{fig:lc}{Lightcurve example. Mrk 421. Figure~from~\citet{lc}. Data not yet available in gamma-cat.}

\section{Conclusions}

\gammacat\ is a TeV gamma-ray astronomy data collection and source catalog. Compared to previous similar projects, \gammacat\ stands out be being fully open-access and set up in a way that makes maintenance and community contributions easy, following the lead of other open astronomy catalogs from \url{https://astrocats.space/}. Merging of some previous TeV data collections (e.g. the light curve archive as DESY) as well as communication about possible collaboration is ongoing. \gammacat\ is being developed in close collaboration with ongoing efforts to develop gamma-ray data format standards (\gadf), open-source science tools (\gammapy) as well as a website dedicated to the gamma-ray sky (\gammasky). \gammacat\ lets you quickly access all published information for all TeV sources, ready for data exploration or analysis. We hope that this will be a useful resource for daily work as well as the basis for novel studies involving the extensive set of available archival gamma-ray data. Concretely we propose \gammacat\ to be used as one of the inputs to build the sky model for the planned CTA data challenge. Feedback and contributions to \gammacat\ are highly welcome!

% \clearpage % To force this stuff to happen by this point in the text, otherwise these will probably end up after the references.

\acknowledgements We thank Fabrizio Lucarelli, Gernot Maier, Konrad Bernl\"ohr and Tarek Hassan for useful discussions or feedback on gamma-cat.
This research has made use of NASA's Astrophysics Data System Bibliographic Services, the SIMBAD database, operated at CDS, Strasbourg, France, the TeVCat online source catalog and the TeGeV catalog at ASDC, as well as the following astronomy Python packages: Astropy, Gammapy, ads.

\bibliography{P6-7}

\end{document}
